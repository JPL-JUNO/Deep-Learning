\documentclass{article}
\usepackage{ctex}
\usepackage{mathptmx}
\usepackage{xcolor}
\definecolor{Claret}{HTML}{990000}
\usepackage[left=2.45cm, right=2.45cm, top=2.77cm, bottom=2.77cm]{geometry}
\usepackage[colorlinks, urlcolor=Claret, linkcolor=orange, linktocpage]{hyperref}
\title{THUDM-chatglm-6b 部署命令}
\author{Stephen CUI}
\date{2023-09-12}
\begin{document}
\maketitle

\tableofcontents

\section{安装 Python 与环境创建}
这个命令可以安装 Python 其他版本,系统带有 Python3.8,可以不安装

\verb|sudo apt-get install python3.10|,其中 3.10 可以换成任意其他的 Python 版本。
% pip --version

\begin{verbatim}
    % 删除默认的软连接,因为默认 python 链接到 python2
    sudo rm -rf /usr/bin/python3
    % 将python 默认设置为 Python3.8,Python3.8 应该是 Ubuntu20 默认安装的版本
    sudo ln -s /usr/bin/python3.8 /usr/bin/python
\end{verbatim}

安装 Python3 的 pip

\verb|sudo apt-get install python3-pip|

可以不运行,除非版本不够(升级至最新的 pip):

\verb|pip install --upgrade pip|

从 Python 3.6 开始,创建虚拟环境的推荐方法是使用 venv 模块。因此我们先要安装提供 venv 模块的 python3-venv 软件包。运行命令:
\begin{verbatim}
    sudo apt install python3-venv
\end{verbatim}

切换项目目录。在目录中,运行
\begin{verbatim}
    python -m venv llm
\end{verbatim}
命令来创建新的虚拟环境,llm 可以更改为任意环境名。

要开始使用此虚拟环境,您需要通过运行 activate 脚本将其激活。source 命令将会加载 python 的虚拟环境
\begin{verbatim}
    source llm/bin/activate
    % conda activate llm
\end{verbatim}

一旦激活,虚拟环境的 bin 目录将添加到 PATH 变量的开头。 此外,您的 Shell 提示符也会更改,并且会显示您当前正在使用的虚拟环境的名称。

使用 Python 包管理器 pip 安装 python 包,在虚拟环境中,可以使用命令 pip 代替 pip3,并使用 python 代替 python3
\begin{verbatim}
    %  测试 numpy 的安装
    pip install numpy  
\end{verbatim}

完成工作后停用虚拟环境,只需键入 deactivate,您将返回到常规 shell。
\begin{verbatim}
    deactivate
\end{verbatim}


\section{下载模型与代码}
\subsection{下载权重}
\subsubsection{方式一}
直接从 \href{https://huggingface.co/}{Hugging Face} 的网站上下载,地址为:\href{https://huggingface.co/THUDM/chatglm2-6b/tree/main}{THUDM/chatglm2-6b}


\subsubsection{手动下载权重的话不需要运行以下步骤}
\begin{verbatim}
    sudo apt-get update
\end{verbatim}

安装 git-lfs 因为较大文件 git 不管理,需要使用 git-lfs 下载,但是下载较慢,建议在\href{https://cloud.tsinghua.edu.cn/d/fb9f16d6dc8f482596c2/}{清华云盘}中下载。

\begin{verbatim}
    sudo apt-get install git-lfs
\end{verbatim}

\subsubsection{方式二(推荐)}
从清华云盘中下载模型权重:地址为:\href{https://cloud.tsinghua.edu.cn/d/fb9f16d6dc8f482596c2/}{chatglm-6b}

\subsection{下载运行代码\label{download}}
注意:\textbf{必须在 Git Bash 中运行}

需要安装 git,下载地址为:\href{https://git-scm.com/}{git}

如果需要,可以设置用户名和邮箱:
\begin{verbatim}
    git config --global user.name "FIRST_NAME LAST_NAME"
    git config --global user.email "MY_NAME@example.com"
\end{verbatim}

如果有代理的话,设置代理的端口(否则连接错误)
\begin{verbatim}
    git config --global http.proxy http://127.0.0.1:port
\end{verbatim}
port 设置为机子的端口。

执行如下命令可以下载代码,代码会保存在当前目录下的 chatglm-6b 文件夹中,可以移动到权重相同文件夹下:

\verb|GIT_LFS_SKIP_SMUDGE=1 git clone https://huggingface.co/THUDM/chatglm-6b|

\section{本地文件上传至服务器}
\subsection{基于 vscode 的拓展 SFTP}
安装完 SFTP 后键盘按 \verb|ctrl+shift+p| 输入 \verb|> SFTP: config| 回车进入 sftp.json 文件。

使用 SFTP 传输本地文件的配置:remotePath,host,password,name 按实际修改,ignore为不传输文件类型或文件
\begin{verbatim}
{
    "name": "ECS-aeqk",
    "host": "IP地址",
    "protocol": "sftp",
    "port": 22,
    "username": "用户名",
    "password": "你的密码",
    "remotePath": "llm/glm",
    "uploadOnSave": true,
    "useTempFile": false,
    "openSsh": false,
    "ignore": [
        "**/.vscode/**",
        "**/.git/**",
        "**/.DS_Store/**",
        "**/__pycache__/**"
    ]
}
\end{verbatim}

\section{安装运行依赖的包}
运行如下代码:
\begin{verbatim}
    pip install -r requirements.txt -i https://pypi.tuna.tsinghua.edu.cn/simple
\end{verbatim}

点击获取 \href{https://github.com/THUDM/ChatGLM2-6B/blob/main/requirements.txt}{requirements.txt} 地址

\subsection{运行测试代码}

可以在 Python 交互式界面中运行以下命令(逐行)(进行交互式页面 python -i)
\begin{verbatim}
from transformers import AutoTokenizer, AutoModel
tokenizer = AutoTokenizer.from_pretrained("./", trust_remote_code=True, revision='v1.1.0')
model = AutoModel.from_pretrained("./", trust_remote_code=True, revision='v1.1.0').half().cuda()
model = model.eval()
response, history = model.chat(tokenizer, "你好", history=[])
print(response)
response, history = model.chat(tokenizer, "如何学习语言大模型", history=history)
print(response)
\end{verbatim}

也可以直接写入文件,比如 \verb|running_test.py|,然后运行
\begin{verbatim}
    python running_test.py
\end{verbatim}

\section{网页版}
\subsection{基于 \href{https://streamlit.io/}{streamlit} 推荐}
安装网页版本所需的包
\begin{verbatim}
    pip install streamlit -i https://pypi.tuna.tsinghua.edu.cn/simple
    pip install streamlit-chat -i https://pypi.tuna.tsinghua.edu.cn/simple
\end{verbatim}

安装完成,直接运行下面命令就可以网页版:

\verb|streamlit run web_demo2.py|

\subsection{基于 \href{https://www.gradio.app/}{gradio} 国内不推荐}
Gradio 是通过友好的 Web 界面演示机器学习模型的最快方式,以便任何人都可以在任何地方使用它!

\verb|pip install gradio mdtex2html|

Gradio 直接使用 Python 运行,国内访问速度慢。
\begin{verbatim}
    python web_deom.py
\end{verbatim}

\section{微调模型}
\subsection{微调所需依赖包}
\verb|pip install rouge_chinese nltk jieba datasets -i https://pypi.tuna.tsinghua.edu.cn/simple|
\subsection{软件依赖}
\href{https://developer.nvidia.com/cuda-toolkit-archive}{cuda 更多版本}

更新 cuda 使用以下命令
\begin{verbatim}
wget https://developer.download.nvidia.com/compute/cuda/11.7.1/local_installers/cuda_11.7.1_515.65.01_linux.run|
\end{verbatim}

\verb|sudo sh cuda_11.7.0_515.43.04_linux.run|

如果你安装了驱动,在安装的时候需要取消驱动的勾选:

\verb|apt install ubuntu-drivers-common|

\verb|ubuntu-drivers devices|

执行微调的文件,用以下命令:

\verb|bash train.sh|

\subsection{微调后的 API}
使用微调后的权重,已经有写好的文件可以运行(\href{https://github.com/THUDM/ChatGLM2-6B/blob/main/ptuning/web_demo.py}{web\_demo.py}),只需要将 \href{https://github.com/THUDM/ChatGLM2-6B/blob/main/ptuning/web_demo.sh}{web\_demo.sh} 文件的参数 \verb|--model_name_or_path| 改成我们本地对应的文件夹或者路径即可。

如果需要打开公网接口,需要将 \verb|demo.queue().launch(share=False, inbrowser=True)| 中的 \verb|share| 参数设置为 \verb|True|,该行位置在 \href{https://github.com/THUDM/ChatGLM2-6B/blob/3d0225f969d56c058f052f6800a21630d14a1184/ptuning/web_demo.py#L162}{web\_demo.py Line 162}.

打开公网接口,会出现一些问题,解决见 \nameref{question}。
\section{其他}
\subsection{常见问题\label{question}}
\begin{itemize}
    \item \href{https://github.com/THUDM/ChatGLM2-6B/issues/21}{RuntimeError: Internal: src/sentencepiece\_processor.cc(1101) [model\_proto-$>$ParseFromArray(serialized.data(), serialized.size())]} 可以重新在 Hugging Face 上下载文件 \nameref{download}。
    \item \href{https://github.com/THUDM/ChatGLM2-6B/issues/208}{RuntimeError: Default process group has not been initialized, please make sure to call init\_process\_group.} 将 transformers 降级至 4.27.1,使用 \verb|pip install transformers==4.27.1|
    \item Could not create share link. Missing file: /xxxx/frpc\_linux\_amd64\_v0.2. 解决步骤如下
          \begin{enumerate}
              \item 下载必要的文件 \href{https://cdn-media.huggingface.co/frpc-gradio-0.2/frpc_linux_amd64}{frpc\_linux\_amd64},会被识别为病毒,需要在防火墙开启权限并执行
              \item 将下载的文件名修改为 frpc\_linux\_amd64\_w0.2
              \item 将文件移动到 此路径 /root/username/lib/pthon3.8/site-packages/gradio 下,username 是你自己的用户,移动命令为

                    \verb|mv frpc_linux_amd64_w0.2 /root/username/lib/pthon3.8/site-packages/gradio|
              \item 给 frpc\_linux\_amd64\_w0.2 添加一个可执行权限

                    \verb|chmod +x /root/username/lib/pthon3.8/site-packages/gradio/frpc_linux_amd64_w0.2|
          \end{enumerate}
\end{itemize}
\subsection{参考链接}
\href{https://www.myfreax.com/how-to-create-python-virtual-environments-on-ubuntu-18-04/}{如何在 Ubuntu 18.04 创建 Python 虚拟环境}

\href{https://github.com/yizhongw/self-instruct/blob/main/human_eval/README.md}{yizhongw/self-instruct}

\href{https://github.com/THUDM/ChatGLM-6B}{清华大学GLM-6B}

\href{https://cloud.tsinghua.edu.cn/d/fb9f16d6dc8f482596c2/}{模型权重所在云盘位置}
\end{document}
