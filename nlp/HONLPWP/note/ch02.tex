\chapter{使用 NLTK 进行文本分类和词性标注\label{ch02}}

自然语言工具包(Natural Language Toolkit,NLTK)是一个用于 NLP 任务的 Python 库,其功能涉及分词、分句、执行进阶任务(如语法分析和文本分类),等等。NLTK 提供了一些针对自然语言的模块和接口,可用于执行诸如文档主题识别、词性标注、情感分析等任务。为了实验各种 NLP 任务,NLTK 还提供了各种文本语料库的模块,从基本的文本集合到带标签的结构化文本(如 WordNet)。
\section{文本预处理及探索性分析}
文本预处理步骤涉及如分词、词干提取和去除停用词之类的任务。对准备好的文本数据进行探索性分析可以了解其主要特征,包括文本的主题和词频分布。
\subsection{分词}
单词词元(token)是任何 NLP 任务都会涉及的文本基本单元。处理文本时,第一步就是将文本拆分为词元。NLTK 为此提供了不同类型的分词器。

为实现基于标点和空格的文本分割,NLTK 也提供了能同时标注出标点符号的 \verb|wordpunct_tokenize| 分词器。

我们也可以使用 NLTK 的正则表达式分词器实现自定义分词。
\subsection{词干提取}
词干提取是一种文本预处理任务,将单词的相关或相似变体(例如 walking)转换为其基本形式(例如 walk),因为它们具有相同的含义。词干提取转换的基本操作之一是将单词的复数形式还原为单数形式,例如将 apples 还原为 apple。尽管这是一个非常简单的转换,但确实存在更加复杂的操作。
\subsection{去除停用词}
常用英文单词(例如 the、is 和 he 等)通常称为停用词。其他语言也有类似的常用词,同属这一类别。去除停用词是 NLP 应用中另一个常见的预处理步骤。在此步骤中,我们将删除那些对文档没有任何意义的词,例如语法冠词和代词。诸如 a、an、he 和 her 等单词都是需要去除的。停用词在整个文本中频繁出现,但它们本身可能不会对 NLP 任务(例如文本分类或搜索)产生任何影响。除英语外,NLTK 还为 21 种语言提供了停用词语料库。
\subsection{探索性分析}
获得词元数据后,常用的基本分析之一是对单词或词元及其在文档中的分布进行计数,从而更多地了解文档中的主要话题。

为了获得文本的频数分布,可以利用 nltk.FreqDist() 函数来获取文本中使用最为频繁的单词。
\section{词性标注}
\subsection{词性标注定义}
词性标注将句子中的单词以不同语义功能或语法功能进行分类。在英语中,主要的词性为名词、代词、形容词、动词、副词、介词、限定词和连词,而词性标注正是为文本中的每个单词或词元附加这些类别之一。NLTK 提供了标注好的文本语料库和一组词性训练器,用以创建自定义的标注器。NLTK 中最常见的标注数据集是 Penn Treebank 和 Brown Corpus,前者由经过分析的期刊日志、电话交谈等文本集合组成,而后者则由 15 种不同类别(科学、政治、宗教和体育等)的文章组成。这些文本数据提供了细粒度标注,然而许多应用可能只需要以下的通用标注集:
\begin{itemize}
    \item VERB:动词(所有时态和方式)
    \item NOUN:名词(普通名词、专有名词)
    \item PRON:代词
    \item ADJ:形容词
    \item ADV:副词
    \item ADP:介词(前置词、后置词)
    \item CONJ:连词
    \item DET:限定词
    \item NUM:基数
    \item PRT:小品词或其他功能词
    \item X-other:外来词、错别字、缩写
    \item .:标点符号
\end{itemize}
NLTK 还提供了从带标注语料库(例如 Brown Corpus)到通用标签的映射。与通用标签集相比,Brown Corpus 的词性标注粒度更细。例如,VBD 标注(用于过去式动词)和 VB 标注(用于基本形式动词)会被映射到通用标注集中的 VERB。
\subsection{词性标注的应用}
词性标注可在命名实体识别(NER)、情感分析、问答和单词消歧中得到应用。NLTK 内置了一个训练好的分类器用以识别文本中的实体。
\section{训练词袋分类器}
词袋作为文本的向量表示,其中每个向量维都能捕获文本中单词的出现频率、存在与否或加权值,但是不能捕获单词之间的顺序。