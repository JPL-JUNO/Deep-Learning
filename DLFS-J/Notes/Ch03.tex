\chapter{神经网络}
关于感知机,既有好消息,也有坏消息。好
消息是,即便对于复杂的函数,感知机也隐含着能够表示它的可能性。上一
章已经介绍过,即便是计算机进行的复杂处理,感知机(理论上)也可以将
其表示出来。坏消息是,设定权重的工作,即确定合适的、能符合预期的输
入与输出的权重,现在还是由人工进行的。
\section{从感知机到神经网络}
\subsection{神经网络的例子}
用图来表示神经网络的话,如\autoref{Examples of Neural Networks}所示。我们把最左边的一列称为
输入层,最右边的一列称为输出层,中间的一列称为中间层。中间层有时也称为隐藏层。“隐藏”一词的意思是,隐藏层的神经元(和输入层、输出
层不同)肉眼看不见。另外,本书中把输入层到输出层依次称为第 0层、第
1 层、第2 层。

\figures{Examples of Neural Networks}
\subsection{复习感知机}
引入新函数$h(x)$,将\autoref{eq2-2}改写成下面的方程:
\begin{equation}
    y=h(b+w_1x_1+w_2x_2)
\end{equation}
\begin{equation}
    \label{eq3-3}
    h(x)=\left\{
    \begin{array}{ll}
        0 & ,x\leq 0 \\
        1 & ,x>0
    \end{array}
    \right.
\end{equation}
\subsection{激活函数登场}
\autoref{eq3-3}中的$h(x)$函数会将输入信号的总和转换为输出信号,这种函数
一般称为激活函数(activation function)。
\begin{tcolorbox}
    本书在使用“感知机”一词时,没有严格统一它所指的算法。一
    般而言,“朴素感知机”是指单层网络,指的是激活函数使用了阶
    跃函数A 的模型。“多层感知机”是指神经网络,即使用sigmoid
    函数等平滑的激活函数的多层网络。
\end{tcolorbox}

\section{激活函数}
\autoref{eq3-3}表示的激活函数以阈值为界,一旦输入超过阈值,就切换输出。
这样的函数称为“阶跃函数\footnote{阶跃函数是指一旦输入超过阈值,就切换输出的函数。应该有更严格的数学定义}”。因此,可以说感知机中使用了阶跃函数作为
激活函数。也就是说,在激活函数的众多候选函数中,感知机使用了阶跃函数。
\subsection{sigmoid函数}
神经网络中经常使用的一个激活函数是sigmoid函数(sigmoid function):
\begin{equation}
    h(x)=\frac{1}{1+\exp(-x)}
\end{equation}
\subsection{sigmoid函数和阶跃函数的比较}
首先注意到的是“平滑性”的不同。sigmoid函数是一条平
滑的曲线,输出随着输入发生连续性的变化。而阶跃函数以0为界,输出发
生急剧性的变化。sigmoid函数的平滑性对神经网络的学习具有重要意义。

另一个不同点是,相对于阶跃函数只能返回0或1,sigmoid函数可以返
回0.731 . . .、0.880 . . .等实数(这一点和刚才的平滑性有关)。也就是说,感
知机中神经元之间流动的是0或1的二元信号,而神经网络中流动的是连续
的实数值信号。

接着说一下阶跃函数和sigmoid函数的共同性质。阶跃函数和sigmoid
函数虽然在平滑性上有差异,可以发现它们
具有相似的形状。实际上,两者的结构均是“输入小时,输出接近0(为0);
随着输入增大,输出向1靠近(变成1)”。也就是说,当输入信号为重要信息时,
阶跃函数和sigmoid函数都会输出较大的值;当输入信号为不重要的信息时,
两者都输出较小的值。还有一个共同点是,不管输入信号有多小,或者有多
大,输出信号的值都在0到1之间。

\subsection{非线性函数}
神经网络的激活函数必须使用非线性函数。换句话说,激活函数不能使
用线性函数。为什么不能使用线性函数呢?因为使用线性函数的话,加深神
经网络的层数就没有意义了。

线性函数的问题在于,不管如何加深层数,总是存在与之等效的“无
隐藏层的神经网络”。为了具体地(稍微直观地)理解这一点,我们来思
考下面这个简单的例子。这里我们考虑把线性函数 $h(x) = cx$ 作为激活
函数,把 $y(x) = h(h(h(x)))$ 的运算对应 3 层神经网络\footnote{该对应只是一个近似,实际的神经网络运算比这个例子要复杂,但不影响后面的结论成立。}。这个运算会进行
$y(x) = c \times c \times c \times x$的乘法运算,但是同样的处理可以由$y(x) = ax$(注意,
$a = c^3$)这一次乘法运算(即没有隐藏层的神经网络)来表示。如本例所示,
使用线性函数时,无法发挥多层网络带来的优势。因此,为了发挥叠加层所
带来的优势,激活函数必须使用非线性函数。

\subsection{ReLU函数}
在神经网络发展的历史上,sigmoid函数很早就开始被使用了,而最近则主要
使用ReLU(Rectified Linear Unit)函数。ReLU函数可以表示为:
\begin{equation}
    h(x)=\left\{
    \begin{array}{ll}
        x & , x>0     \\
        0 & , x\leq 0 \\
    \end{array}
    \right.
\end{equation}
\section{多维数组的运算}
\verb|np.dot()|接收两个NumPy数组作为参数,并返回数组的乘积。

\section{3层神经网络的实现}
\figures{Signal passing from input layer to layer 1}
\autoref{Signal passing from input layer to layer 1}中增加了表示偏置的神经元“1”。请注意,偏置的右下角的索引
号只有一个。这是因为前一层的偏置神经元(神经元“1”)只有一个\footnote{任何前一层的偏置神经元“1”都只有一个。偏置权重的数量取决于后一层的神经元的数量(不包括
    后一层的偏置神经元“1”)。偏置是在后一层的神经元上增加,每个样本增加的偏置是相同,但是分量不一样。}。

\figures{Layer 1 to Layer 2 Signaling}
\figures{Signal passing from layer 2 to output layer}
\section{输出层的设计}
神经网络可以用在分类问题和回归问题上,不过需要根据情况改变输出
层的激活函数。一般而言,回归问题用恒等函数,分类问题用softmax函数。
\subsection{恒等函数和softmax函数}
恒等函数会将输入按原样输出,对于输入的信息,不加以任何改动地直
接输出。因此,在输出层使用恒等函数时,输入信号会原封不动地被输出。和前面介绍的隐藏层的激活函数一样,恒等函数进行的转换处理可以
用一根箭头来表示。

分类问题中使用的softmax函数表示为:
\begin{equation}
    \label{eq3-10}
    y_k=\frac{\exp(a_k)}{\sum\limits_{i=1}^n{\exp(a_i)}}
\end{equation}

softmax函数的分子是输入信号$a_k$的指数函数,分母是所有输入信号的指数
函数的和。

\subsection{实现softmax函数时的注意事项}
\autoref{eq3-10}在计算机的运算
上有一定的缺陷。这个缺陷就是溢出问题。softmax函数的实现中要进行指
数函数的运算,但是此时指数函数的值很容易变得非常大。

\begin{equation}
    \begin{aligned}
        y_k=\frac{\exp(a_k)}{\sum\limits_{i=1}^n{\exp(a_i)}} & =\frac{C\exp(a_k)}{C\sum\limits_{i=1}^n{\exp(a_i)}}             \\
                                                             & =\frac{\exp(a_k+\log C)}{\sum\limits_{i=1}^n{\exp(a_i+\log C)}} \\
                                                             & =\frac{\exp(a_k+C^{'})}{\sum\limits_{i=1}^n{\exp(a_i+C^{'})}}   \\
    \end{aligned}
\end{equation}

在进行softmax的指数函数的运算时,加上(或者减去)
某个常数并不会改变运算的结果。这里的$C^{'}$可以使用任何值,但是为了防
止溢出,一般会使用输入信号中的最大值。
\subsection{softmax函数的特征}
一般而言,神经网络只把输出值最大的神经元所对应的类别作为识别结果。
并且,即便使用softmax函数,输出值最大的神经元的位置也不会变。因此,
神经网络在进行分类时,输出层的softmax函数可以省略。在实际的问题中,
由于指数函数的运算需要一定的计算机运算量,因此输出层的softmax函数
一般会被省略。
\subsection{输出层的神经元数量}
输出层的神经元数量需要根据待解决的问题来决定。对于分类问题,输
出层的神经元数量一般设定为类别的数量。
\section{手写数字识别}
假设学习已经全部结束,我们使用学习到的参
数,先实现神经网络的“推理处理”。这个推理处理也称为神经网络的\textbf{前向
    传播}(forward propagation)。

\subsection{批处理}
\figures{change of array shape}

从整体的处理流程来看,\autoref{change of array shape}中,输入一个由784个元素(原本是一
个$28\times 28$的二维数组)构成的一维数组后,输出一个有10个元素的一维数组。
这是只输入一张图像数据时的处理流程。

现在我们来考虑打包输入多张图像的情形。这种打包式的输入数据称为\textbf{批}(batch)。批有“捆”的意思,图像就如同
纸币一样扎成一捆。

\begin{tcolorbox}
    批处理对计算机的运算大有利处,可以大幅缩短每张图像的处理时
    间。那么为什么批处理可以缩短处理时间呢?这是因为大多数处理
    数值计算的库都进行了能够高效处理大型数组运算的最优化。并且,
    在神经网络的运算中,当数据传送成为瓶颈时,批处理可以减轻数
    据总线的负荷(严格地讲,相对于数据读入,可以将更多的时间用在
    计算上)。也就是说,批处理一次性计算大型数组要比分开逐步计算
    各个小型数组速度更快。
\end{tcolorbox}

\figures{Variation of array shape in batch}

\section{小结}
\begin{itemize}
    \item 神经网络中的激活函数使用平滑变化的sigmoid函数或ReLU函数。
    \item 通过巧妙地使用NumPy多维数组,可以高效地实现神经网络。
    \item 机器学习的问题大体上可以分为回归问题和分类问题。
    \item 关于输出层的激活函数,回归问题中一般用恒等函数,分类问题中一般用softmax函数。
    \item 分类问题中,输出层的神经元的数量设置为要分类的类别数。
    \item 输入数据的集合称为批。通过以批为单位进行推理处理,能够实现高速的运算。
\end{itemize}