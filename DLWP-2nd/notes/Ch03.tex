\chapter{Introduction to Keras and TensorFlow\label{Ch03}}

\section{神经网络剖析}
\figures{fig3-1}{多个层链接在一起组成了网络,将输入数据映射为预测值。然后损失函数将这些预测值与目标进行比较,得到损失值,用于衡量网络预测值与预期结果的匹配程度。优化器使用这个损失值来更新网络的权重。}

\subsection{层:深度学习的基础组件}
神经网络的基本数据结构是层。层是一个数据处理模块,将一个或多个输入张量转换为一个或多个输出张量。有些层是无状态的,但大多数的层是有状态的,即层的权重。权重是利用随机梯度下降学到的一个或多个张量,其中包含网络的知识。不同的张量格式与不同的数据处理类型需要用到不同的层。例如,简单的向量数据保存在形状为 (samples, features) 的 2D 张量中,通常用密集连接层[densely connected layer,也叫全连接层(fully connected layer)或密集层(dense layer),对应于 Keras 的 Dense 类]来处理。序列数据保存在形状为 (samples, timesteps, features) 的 3D 张量中,通常用循环层(recurrent layer,比如 Keras 的 LSTM 层)来处理。图像数据保存在 4D 张量中,通常用二维卷积层(Keras 的 Conv2D)来处理。

层兼容性(layer compatibility)具体指的是每一层只接受特定形状的输入张量,并返回特定形状的输出张量。


\section{First steps with TensorFlow}
\subsection{Constant tensors and variables}
A significant difference between NumPy arrays and TensorFlow tensors is that TensorFlow tensors aren't assignable: they're constant.

To train a model, we'll need to update its state, which is a set of tensors. If tensors aren't assignable, how do we do it? That's where variables come in. \textsf{tf.Variable} is the class meant to manage modifiable state in TensorFlow.

Similarly, assign\_add() and assign\_sub() are efficient equivalents of += and -=.

\subsubsection*{THE BASE LAYER CLASS IN KERAS}
A simple API should have a single abstraction around which everything is centered. In Keras, that's the Layer class. Everything in Keras is either a Layer or something that closely interacts with a Layer.

A Layer is an object that encapsulates some state (weights) and some computation (a forward pass). The weights are typically defined in a build() (although they could also be created in the constructor, \_\_init\_\_()), and the computation is defined in the call() method.