\chapter{\label{Ch04}}
\section{新闻分类:多分类问题}
\subsection{小结}
如果要对 $N$ 个类别的数据点进行分类,网络的最后一层应该是大小为 $N$ 的 Dense 层。
\begin{itemize}
    \item 对于单标签、多分类问题,网络的最后一层应该使用 softmax 激活,这样可以输出在 $N$ 个输出类别上的概率分布。
    \item  这种问题的损失函数几乎总是应该使用分类交叉熵。它将网络输出的概率分布与目标的真实分布之间的距离最小化。
    \item  处理多分类问题的标签有两种方法:
          \begin{enumerate}
              \item  通过分类编码(也叫 one-hot 编码)对标签进行编码,然后使用 \verb|categorical_crossentropy| 作为损失函数。
              \item  将标签编码为整数,然后使用 \verb|sparse_categorical_crossentropy| 损失函数。
          \end{enumerate}
    \item 如果你需要将数据划分到许多类别中,应该避免使用太小的中间层,以免在网络中造成信息瓶颈。
\end{itemize}

\section{预测房价:回归问题}
\subsection{小结}
\begin{itemize}
    \item 回归问题使用的损失函数与分类问题不同。回归常用的损失函数是均方误差(MSE)。
    \item 同样,回归问题使用的评估指标也与分类问题不同。显而易见,精度的概念不适用于回归问题。常见的回归指标是平均绝对误差(MAE)。
    \item 如果输入数据的特征具有不同的取值范围,应该先进行预处理,对每个特征单独进行缩放。
    \item 如果可用的数据很少,使用 $K$ 折验证可以可靠地评估模型。
    \item 如果可用的训练数据很少,最好使用隐藏层较少(通常只有一到两个)的小型网络,以避免严重的过拟合。
\end{itemize}