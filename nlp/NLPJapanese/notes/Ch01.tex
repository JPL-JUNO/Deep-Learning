\chapter{神经网络的复习\label{Ch01}}
\section{神经网络的学习}
\subsection{损失函数}
我们来介绍一下 Softmax 函数和交叉熵误差。首先,Softmax 函数可由下式表示:
\begin{equation}
    y_k=\frac{\exp{s_k}}{\sum_{i=1}^{n}\exp{s_i}}
\end{equation}

此时,交叉熵误差可由下式表示:
\begin{equation}
    L=-\sum_{k}t_k\log{y_k}
\end{equation}
这里,$t_k$ 是对应于第 $k$ 个类别的监督标签。监督标签以 one-hot 向量的形式表示,比如 $\bm{t} = (0, 0, 1)$。

另外,在考虑了 mini-batch 处理的情况下,交叉熵误差可以由下式表示:
\begin{equation}
    L=-\frac{1}{N}\sum_{n}\sum_{k}t_{nk}\log{y_{nk}}
\end{equation}
这里假设数据有 $N$ 笔,$t_{nk}$ 表示第 $n$ 笔数据的第 $k$ 维元素的值,$y_{nk}$ 表示神经网络的输出,$t_{nk}$ 表示监督标签。通过这样的平均化,无论 mini-batch 的大小如何,都始终可以获得一致的指标。
\subsection{梯度与导数}
\begin{tcolorbox}
    严格地说,这里使用的“梯度”一词与数学中的“梯度”是不同的。数学中的梯度仅限于关于向量的导数。而在深度学习领域,一般也会定义关于矩阵和张量的导数,称为“梯度”。
\end{tcolorbox}
\subsection{链式法则}
链式法则的重要之处在于,无论我们要处理的函数有多复杂(无论复合了多少个函数),都可以根据它们各自的导数来求复合函数的导数。也就是说,只要能够计算各个函数的局部的导数,就能基于它们的积计算最终的整体的导数。
\figures{fig1-17}{计算图的反向传播}
\subsubsection*{分支节点}
\figures{fig1-20}{分支节点的正向传播(左图)和反向传播(右图)}
严格来说,分支节点并没有节点,只有两根分开的线。此时,相同的值被复制并分叉。因此,分支节点也称为复制节点。如 \autoref{fig1-20} 所示,它的反向传播是上游传来的梯度之和。
\subsubsection*{Repeat 节点}
分支节点有两个分支,但也可以扩展为 $N$ 个分支(副本),这里称为 Repeat 节点。现在,我们尝试用计算图绘制一个 Repeat 节点(图 \autoref{fig1-21})。
\figures{fig1-21}{Repeat 节点的正向传播(上图)和反向传播(下图)}
\subsection*{Sum 节点}
Sum 节点是通用的加法节点。这里考虑对一个 $N \times D$ 的数组沿第 0 个轴求和。此时,Sum 节点的正向传播和反向传播如图 \autoref{fig1-22} 所示。 有趣的是,Sum 节点和 Repeat 节点存在逆向关系。所谓逆向关系,是指 Sum 节点的正向传播相当于 Repeat 节点的反向传播,Sum 节点的反向传播相当于 Repeat 节点的正向传播。
\figures{fig1-22}{Sum 节点的正向传播(上图)和反向传播(下图)}
\subsubsection*{MatMul 节点}
考虑 $\bm{y}=\bm{x}\bm{W}$ 这个计算,这里,$\bm{x}$、
$\bm{W}$、$\bm{y}$ 的形状分别是 $1 \times D$、$D \times H$、$1 \times H$。此时,可以按如下方式求得关于 $\bm{x}$ 的第 $i$ 个元素的导数 $\frac{\partial L}{\partial x_i}$。
\begin{equation}
    \label{eq1.12}
    \frac{\partial L}{\partial x_i}=\sum_{j}\frac{\partial L}{\partial y_j}\frac{\partial y_j}{\partial x_i}
\end{equation}
利用 $\frac{\partial y_j}{\partial x_i}=W_{ij}$,将其代入 \autoref{eq1.12}:
\begin{equation}
    \frac{\partial L}{\partial x_i}=\sum_{j}\frac{\partial L}{\partial y_j}\frac{\partial y_j}{\partial x_i}=\sum_{j}\frac{\partial L}{\partial y_j}W_{ij}
\end{equation}
$\frac{\partial L}{\partial x_i}$ 由向量 $\frac{\partial L}{\partial \bm{y}}$ 和 $\bm{W}$ 的第 $i$ 行向量的内积求得。从这个关系可以导出下式:
\begin{equation}
    \frac{\partial L}{\partial \bm{x}}=\frac{\partial L}{\partial \bm{y}}\bm{W}^T
\end{equation}

和省略号一样,这里也可以进行基于 \verb|grads[0] = dW| 的赋值。不同的是,在使用省略号的情况下会覆盖掉 NumPy 数组。这是浅复制(shallow copy)和深复制(deep copy)的差异。\verb|grads[0] = dW| 的赋值相当于浅复制,\verb|grads[0][...] = dW| 的覆盖相当于深复制。

\figures{fig1-26}{通过确认矩阵形状,推导反向传播的数学式}

浅复制中,\verb|a| 的指向发生改变(指向的内存地址改变),而深复制中 \verb|a| 的指向没有发生改变,只是指向的内存地址内的值改变(\autoref{fig1-27})。
\figures{fig1-27}{\texttt{a = b} 和 \texttt{a[...] = b} 的区别:使用省略号时数据被覆盖,变量指向的内存地址不变}