\documentclass{article}
\usepackage{ctex}
\title{共现矩阵(co-occurrence matrix)}
\author{Stephen CUI}
\date{\today}
\begin{document}
\maketitle
考虑如何基于分布式假设使用向量表示单词,最直截了当的实现方法是对周围单词的数量进行计数。具体来说,在关注某个单词的情况下,对它的周围出现了多少次什么单词进行计数,然后再汇总。这里,我们将这种做法称为“基于计数的方法”,在有的文献中也称为“基于统计的方法”。

考虑下面的句子:You say goodbye and I say hello. 我们将窗口大小设为 1.

\begin{table}[h]
    \centering
    \caption{单词 you 的上下文中包含的单词的频数}
    \begin{tabular}{rrrrrrrr}
        \hline
            & you & say & goodbye & and & i & hello & . \\
        \hline
        you & 0   & 1   & 0       & 0   & 0 & 0     & 0 \\
        \hline
    \end{tabular}
\end{table}

对其他单词做同样的处理,可以得到:
\begin{table}[h]
    \centering
    \caption{汇总各个单词的上下文中包含的单词的频数}
    \begin{tabular}{rrrrrrrr}
        \hline
                & you & say & goodbye & and & i & hello & . \\
        \hline
        you     & 0   & 1   & 0       & 0   & 0 & 0     & 0 \\
        say     & 1   & 0   & 1       & 0   & 1 & 1     & 0 \\
        goodbye & 0   & 1   & 0       & 1   & 0 & 0     & 0 \\
        and     & 0   & 0   & 1       & 0   & 1 & 0     & 0 \\
        i       & 0   & 1   & 0       & 1   & 0 & 0     & 0 \\
        hello   & 0   & 1   & 0       & 0   & 0 & 0     & 1 \\
        .       & 0   & 0   & 0       & 0   & 0 & 1     & 0 \\
        \hline
    \end{tabular}
\end{table}

这个表格的各行对应相应单词的向量。因为表格呈矩阵状,所以称为\textbf{共现矩阵(co-occurrence matrix)}。
\end{document}