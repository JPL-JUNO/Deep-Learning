\chapter{神经网络的数学基础\label{Ch02}}
\section{初识神经网络}
神经网络的核心组件是\textbf{层}(layer),它是一种数据处理模块,你可以将它看成数据过滤器。具体来说,层从输入数据中提取表示——我们期望这种表示有助于解决手头的问题。大多数深度学习都是将简单的层链接起来,从而实现渐进式的\textbf{数据蒸馏}(data distillation)。深度学习模型就像是数据处理的筛子,包含一系列越来越精细的数据过滤器(即层)。

要想训练网络,我们还需要选择\textbf{编译}(compile)步骤的三个参数。
\begin{itemize}
    \item 损失函数(loss function):网络如何衡量在训练数据上的性能,即网络如何朝着正确的方向前进。
    \item 优化器(optimizer):基于训练数据和损失函数来更新网络的机制。
    \item 在训练和测试过程中需要监控的指标(metric)。
\end{itemize}

\section{神经网络的数据表示}
\subsection{关键属性}

张量是由以下三个关键属性来定义的:
\begin{itemize}
    \item 轴的个数(阶)。例如,3D 张量有 3 个轴,矩阵有 2 个轴。这在 Numpy 等 Python 库中也叫张量的 ndim。
    \item 形状。这是一个整数元组,表示张量沿每个轴的维度大小(元素个数)。例如,前面矩阵示例的形状为 (3, 5),3D 张量示例的形状为 (3, 3, 5)。向量的形状只包含一个元素,比如 (5,),而标量的形状为空,即 ()。
    \item 数据类型(在 Python 库中通常叫作 dtype)。这是张量中所包含数据的类型,例如,张量的类型可以是 float32、uint8、float64 等。在极少数情况下,你可能会遇到字符(char)张量。注意,Numpy(以及大多数其他库)中不存在字符串张量,因为张量存储在预先分配的连续内存段中,而字符串的长度是可变的,无法用这种方式存储。
\end{itemize}

\subsection{数据批量的概念}
通常来说,深度学习中所有数据张量的第一个轴(0 轴,因为索引从 0 开始)都是样本轴(samples axis,有时也叫样本维度)。

此外,深度学习模型不会同时处理整个数据集,而是将数据拆分成小批量。

\subsection{现实世界中的数据张量}
我们用几个你未来会遇到的示例来具体介绍数据张量。你需要处理的数据几乎总是以下类
别之一。
\begin{itemize}
    \item 向量数据:2D 张量,形状为 (samples, features)。
    \item  时间序列数据或序列数据:3D 张量,形状为 (samples, timesteps, features)。
    \item  图像:4D 张量,形状为 (samples, height, width, channels) 或 (samples, channels, height, width)。
    \item  视频:5D 张量,形状为 (samples, frames, height, width, channels) 或 (samples, frames, channels, height, width)。
\end{itemize}

\subsubsection*{图像数据}
图像通常具有三个维度:高度、宽度和颜色深度。虽然灰度图像(比如 MNIST 数字图像)只有一个颜色通道,因此可以保存在 2D 张量中,但按照惯例,图像张量始终都是 3D 张量,灰度图像的彩色通道只有一维。

图像张量的形状有两种约定:\textbf{通道在后}(channels-last)的约定(在 TensorFlow 中使用)和\textbf{通道在前}(channels-first)的约定(在 Theano 中使用)。Google 的 TensorFlow 机器学习框架将颜色深度轴放在最后:(samples, height, width, color\_depth)。与此相反,Theano将图像深度轴放在批量轴之后:(samples, color\_depth, height, width)。Keras 框架同时支持这两种格式。
