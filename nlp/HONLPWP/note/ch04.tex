\chapter{使用浅层模型进行语义嵌入\label{ch04}}
\section{词向量}
词向量(word vector)是许多应用中非常有用的构建模块。它可以对词间的语义关系进行捕获和编码,并最终将单词转换为数字序列,从而形成非常适合训练深度学习模型的密集向量。
\subsection{经典方法}
构建单词表示的传统方法一般使用词袋模型。在该模型中,词表示将各个单词视为彼此独立的。因此此类表示通常使用独热编码生成句子或文档的向量表示,以显示句子中单词的存在与否。但这种表示在实际应用中鲜有使用,因为单词的含义会根据周围单词而变化。在使用词袋模型(其中,单词以其自身维度进行编码)的经典方法中,实际上无法对这种语义上的相似性进行编码。

传统方法的另一个不足是无法体现单词在句子中出现的顺序。传统的词袋方法统计文档中文本的词汇量,以获得存在单词的表示形式,但这丢了失上下文。与前面讨论的编码类似,它假定文档中的单词彼此独立。这种方法还有一个局限:会导致数据稀疏,使得统计模型的训练变得更加困难。
\subsection{Word2vec}
单词的向量表示可以实现语义相似单词的连续表示,其中相关的单词会被映射到高维空间内彼此靠近的点上。这种单词表示方法基于以下事实:有相似上下文的单词也有相似的语义。Word2vec 就是这样的一种模型,它试图通过使用相邻的单词来直接预测单词并学习小且密集的向量(也称为嵌入)。Word2vec 可从原始文本中学习词嵌入,是一种在计算上很有效率的无监督模型。为了学习这些密集向量,Word2vec 有两种形式:连续词袋(CBOW)模型和跳字(skip-gram)模型(由 Mikilov 等提出)。

\textbf{Word2vec 是一个浅层的三层神经网络,其中第一层和最后一层构成输入和输出,中间层构建潜在表示以便将输入单词转换为输出向量表示形式}。

Word2vec 单词表示法可以探索词向量之间有趣的数学关系,这也是单词的一种直观表达。
\figures{fig4-1}{表明词向量是如何从 woman 转换到 queen 的,这与从 man 转换到 king 具有相似之处。使用 Word2vec 可以理解该关系,该模型使用了一个简单的三层神经网络来预测周围的单词(给定输入单词)或预测该单词(给定周围的单词)。这两种方法都是 Word2vec 的变体,其中使用输入单词来预测周围单词的方法是跳字模型,而使用周围单词来预测目标单词是连续词袋模型。}
\subsection{连续词袋模型}
Word2vec 的连续词袋模型从一组输入源的上下文单词来预测目标单词。
\subsection{跳字模型}
跳字模型执行连续词袋任务的逆操作,通过使用目标单词来预测上下文中的相邻单词。
\figures{fig4-2}{跳字给定目标单词来预测上下文,而连续词袋会根据目标单词周围固定大小窗口的单词(以词袋表示)来学习预测目标单词。(这个图似乎不太好理解)}

通常,当数据集更大时,跳字方法倾向于产生更好的单词表示。