\chapter{深度学习}
深度学习是加深了层的深度神经网络。基于之前介绍的网络,只需通过
叠加层,就可以创建深度网络。
\section{加深网络}

\subsection{向更深的网络出发}

这里我们来创建一个如 \autoref{Deep CNN for Handwritten Digit Recognition} 所示的网络结构的CNN。

\figures{Deep CNN for Handwritten Digit Recognition}

\subsection{进一步提高识别精度}
在一个标题为“\href{https://rodrigob.github.io/are_we_there_yet/build/classification_datasets_results.html}{What is the class of this image ?}”的网站
上,以排行
榜的形式刊登了目前为止通过论文等渠道发表的针对各种数据集的方法的识
别精度。
\figures{Examples of misidentified images}
排行榜中前几名的方法,可以发现进一步提高识别精度的技术和
线索。比如,集成学习、学习率衰减、\textbf{Data Augmentation}(数据扩充)等都有
助于提高识别精度。尤其是Data Augmentation,虽然方法很简单,但在提高
识别精度上效果显著。

Data Augmentation基于算法“人为地”扩充输入图像(训练图像)。具
体地说,如 \autoref{Data Augmentation example} 所示,对于输入图像,通过施加旋转、垂直或水平方向上
的移动等微小变化,增加图像的数量。这在数据集的图像数量有限时尤其有效。

除了如 \autoref{Data Augmentation example} 所示的变形之外,Data Augmentation还可以通过其他各
种方法扩充图像,比如裁剪图像的 “crop处理”、将图像左右翻转的“fl ip处
理”\footnote{flip处理只在不需要考虑图像对称性的情况下有效。}等。对于一般的图像,施加亮度等外观上的变化、放大缩小等尺度上
的变化也是有效的。不管怎样,通过Data Augmentation巧妙地增加训练图像,
就可以提高深度学习的识别精度。虽然这个看上去只是一个简单的技巧,不
过经常会有很好的效果。

\figures{Data Augmentation example}

\subsection{加深层的动机}
关于加深层的重要性,现状是理论研究还不够透彻。尽管目前相关理论
还比较贫乏,但是有几点可以从过往的研究和实验中得以解释。

首先,从以ILSVRC为代表的大规模图像识别的比赛结果中可以看出加
深层的重要性(详细内容请参考下一节)。这种比赛的结果显示,最近前几名
的方法多是基于深度学习的,并且有逐渐加深网络的层的趋势。也就是说,
可以看到层越深,识别性能也越高。
下面我们说一下加深层的好处。其中一个好处就是可以减少网络的参数
数量。说得详细一点,就是与没有加深层的网络相比,加深了层的网络可以
用更少的参数达到同等水平(或者更强)的表现力。

\figures{5-5 convolution example}

显然,在 \autoref{5-5 convolution example} 的例子中,每个输出节点都是从输入数据的某个
$5\times 5$的区域算出来的。接下来我们思考一下 \autoref{Example of a convolutional layer repeated twice 3-3} 中重复两次$5\times 5$的卷积
运算的情形。此时,每个输出节点将由中间数据的某个$3\times 3$的区域计算出来。

\figures{Example of a convolutional layer repeated twice 3-3}

一次$5\times 5$的卷积运算的区域可以由两次$3\times 3$的卷积运算抵充。并且,
相对于前者的参数数量25($5\times 5$),后者一共是18( $2\times3\times 3$),通过叠加卷
积层,参数数量减少了。而且,这个参数数量之差会随着层的加深而变大。
比如,重复三次$3\times 3$的卷积运算时,参数的数量总共是27。而为了用一次
卷积运算“观察”与之相同的区域,需要一个$7 \times 7$的滤波器,此时的参数数
量是49。

\begin{tcolorbox}
    叠加小型滤波器来加深网络的好处是可以减少参数的数量,扩大\uuline{感受野}(receptive field,给神经元施加变化的某个局部空间区域)。并且,
    通过叠加层,将ReLU等激活函数夹在卷积层的中间,进一步提高
    了网络的表现力。这是因为向网络添加了基于激活函数的“非线性”
    表现力,通过非线性函数的叠加,可以表现更加复杂的东西。
\end{tcolorbox}

加深层的另一个好处就是使学习更加高效。与没有加深层的网络相比,
通过加深层,可以减少学习数据,从而高效地进行学习。

\section{深度学习的小历史}
\subsection{VGG}
VGG 是由卷积层和池化层构成的基础的 CNN。不过,如图 8-9所示,
它的特点在于将有权重的层(卷积层或者全连接层)叠加至16层(或者19层),
具备了深度(根据层的深度,有时也称为“VGG16”或“VGG19”)。

VGG中需要注意的地方是,基于 $3\times 3$ 的小型滤波器的卷积层的运算是
连续进行的。如 \autoref{VGG} 所示,重复进行“卷积层重叠2次到4次,再通过池化
层将大小减半”的处理,最后经由全连接层输出结果。

\figures{VGG}

\subsection{GoogLeNet}
只看 \autoref{GoogLeNet} 的话,这似乎是一个看上去非常复杂的网络结构,但实际上它基
本上和之前介绍的CNN结构相同。不过,GoogLeNet的特征是,网络不仅
在纵向上有深度,在横向上也有深度(广度)。

GoogLeNet在横向上有“宽度”,这称为“Inception结构”。

\figures{GoogLeNet}

\autoref{Inception structure of GoogLeNet},Inception结构使用了多个大小不同的滤波器(和池化),
最后再合并它们的结果。GoogLeNet的特征就是将这个Inception结构用作
一个构件(构成元素)。此外,在GoogLeNet中,很多地方都使用了大小为 $1 \times 1$ 的滤波器的卷积层。这个$1 \times 1$的卷积运算通过在通道方向上减小大小,
有助于减少参数和实现高速化处理。

\figures{Inception structure of GoogLeNet}

\subsection{ResNet}
ResNet是微软团队开发的网络。它的特征在于具有比以前的网络更
深的结构。

我们已经知道加深层对于提升性能很重要。但是,在深度学习中,过度
加深层的话,很多情况下学习将不能顺利进行,导致最终性能不佳。ResNet中,
为了解决这类问题,导入了“快捷结构”
(也称为“捷径”或“小路”)。导入这
个快捷结构后,就可以随着层的加深而不断提高性能了(当然,层的加深也
是有限度的)。

如 \autoref{Components of ResNet} 所示,快捷结构横跨(跳过)了输入数据的卷积层,将输入 x合
计到输出。
\autoref{Components of ResNet} 中,在连续2层的卷积层中,将输入$x$跳着连接至2层后的输出。
这里的重点是,通过快捷结构,原来的2层卷积层的输出$\mathcal{F}(x)$变成了$\mathcal{F}(x) + x$。
通过引入这种快捷结构,即使加深层,也能高效地学习。这是因为,通过快
捷结构,反向传播时信号可以无衰减地传递。

\figures{Components of ResNet}

\begin{tcolorbox}
    因为快捷结构只是原封不动地传递输入数据,所以反向传播时会将
    来自上游的梯度原封不动地传向下游。这里的重点是不对来自上游
    的梯度进行任何处理,将其原封不动地传向下游。因此,基于快捷
    结构,不用担心梯度会变小(或变大),能够向前一层传递“有意义
    的梯度”。通过这个快捷结构,之前因为加深层而导致的梯度变小的
    梯度消失问题就有望得到缓解。
\end{tcolorbox}

\begin{tcolorbox}
    实践中经常会灵活应用使用ImageNet这个巨大的数据集学习到的权
    重数据,这称为\textbf{迁移学习},将学习完的权重(的一部分)复制到其他
    神经网络,进行再学习(fine tuning)。比如,准备一个和VGG相同
    结构的网络,把学习完的权重作为初始值,以新数据集为对象,进
    行再学习。迁移学习在手头数据集较少时非常有效。
\end{tcolorbox}

\section{深度学习的高速化}
\subsection{需要努力解决的问题}
\subsection{基于GPU的高速化}
GPU计算,是指基于GPU进行通用的数值计算的操作。

深度学习中需要进行大量的乘积累加运算(或者大型矩阵的乘积运算)。
这种大量的并行运算正是GPU所擅长的(反过来说,CPU比较擅长连续的、
复杂的计算)。
\subsection{分布式学习}
\subsection{运算精度的位数缩减}
\section{深度学习的应用案例}
\subsection{物体检测}
物体检测是从图像中确定物体的位置,并进行分类的问题。

对于这样的物体检测问题,人们提出了多个基于CNN的方法。这些方
法展示了非常优异的性能,并且证明了在物体检测的问题上,深度学习是非
常有效的。

在使用CNN进行物体检测的方法中,有一个叫作R-CNN的有名的方
法。\autoref{Processing flow of R-CNN} 显示了R-CNN的处理流。
\figures{Processing flow of R-CNN}
\subsection{图像分割}
图像分割是指在像素水平上对图像进行分类。使用以像
素为单位对各个对象分别着色的监督数据进行学习。然后,在推理时,对输
入图像的所有像素进行分类。

要基于神经网络进行图像分割,最简单的方法是以所有像素为对象,对
每个像素执行推理处理。比如,准备一个对某个矩形区域中心的像素进行分
类的网络,以所有像素为对象执行推理处理。正如大家能想到的,这样的
方法需要按照像素数量进行相应次 forward处理,因而需要耗费大量的时间
(正确地说,卷积运算中会发生重复计算很多区域的无意义的计算)。为了解
决这个无意义的计算问题,有人提出了一个名为FCN(Fully Convolutional
Network)的方法。该方法通过一次 forward处理,对所有像素进行分类。

FCN的字面意思是 “全部由卷积层构成的网络”。\important{相对于一般的CNN包含全连接层,FCN将全连接层替换成发挥相同作用的卷积层}。

\begin{tcolorbox}
    全连接层中,输出和全部的输入相连。使用卷积层也可以实现与此
    结构完全相同的连接。比如,针对输入大小是 $32\times10\times10$(通道
    数32、高10、长10)的数据的全连接层可以替换成滤波器大小为
    $32\times10\times10$的卷积层。如果全连接层的输出节点数是100,那么在
    卷积层准备100个$32\times10\times10$的滤波器就可以实现完全相同的处理。
    像这样,全连接层可以替换成进行相同处理的卷积层。
\end{tcolorbox}
\subsection{图像标题生成}
一个基于深度学习生成图像标题的代表性方法是被称为 NIC\marginpar[NIC]{NIC}(Neural
Image Caption)的模型。如 \autoref{Neural Image Caption} 所示,NIC由深层的CNN和处理自然语
言的RNN(Recurrent Neural Network)构成。RNN是呈递归式连接的网络,
经常被用于自然语言、时间序列数据等连续性的数据上。

\figures{Neural Image Caption}

NIC基于CNN从图像中提取特征,并将这个特征传给 RNN。RNN以
CNN提取出的特征为初始值,递归地生成文本。基本上NIC是组合了两个神经网络(CNN和RNN)的简单
结构。基于NIC,可以生成惊人的高精度的图像标题。我们将组合图像和自
然语言等多种信息进行的处理称为多模态处理\marginpar[多模态处理]{多模态处理}。

\begin{tcolorbox}
    RNN的R表示Recurrent(递归的)。这个递归指的是神经网络的递归
    的网络结构。根据这个递归结构,神经网络会受到之前生成的信息
    的影响(换句话说,会记忆过去的信息),这是RNN的特征。比如,
    生成“我”这个词之后,下一个要生成的词受到“我”这个词的影响,
    生成了“要”;然后,再受到前面生成的“我要”的影响,生成了“睡觉”
    这个词。对于自然语言、时间序列数据等连续性的数据,RNN以记
    忆过去的信息的方式运行。
\end{tcolorbox}
\section{深度学习的未来}
\subsection{图像风格变换}
输入两个图像后,会生成一个新的图像。两个输入图像中,一个称为“内容
图像”,另一个称为“风格图像”。此项研究出自论文“\href{https://arxiv.org/abs/1508.06576}{A Neural Algorithm of Artistic Style}”,在学习过程中使网络的中间数据近似内容图像的中间
数据。这样一来,就可以使输入图像近似内容图像的形状。此外,为了从风
格图像中吸收风格,导入了风格矩阵的概念。通过在学习过程中减小风格矩
阵的偏差。
\subsection{图像的生成}
能画出以假乱真的图像的DCGAN\marginpar[DCGAN]{DCGAN}会将图像的生成过程模型化。使用大
量图像(比如,印有卧室的大量图像)训练这个模型,学习结束后,使用这
个模型,就可以生成新的图像。

DCGAN中使用了深度学习,其技术要点是使用了Generator(生成者)
和 Discriminator(识别者)这两个神经网络。Generator 生成近似真品的图
像,Discriminator判别它是不是真图像(是Generator生成的图像还是实际
拍摄的图像)。像这样,通过让两者以竞争的方式学习,Generator会学习到
更加精妙的图像作假技术,Discriminator则会成长为能以更高精度辨别真假
的鉴定师。两者互相切磋、共同成长,这是 GAN(Generative Adversarial
Network)这个技术的有趣之处。在这样的切磋中成长起来的Generator最终
会掌握画出足以以假乱真的图像的能力(或者说有这样的可能)。
\subsection{自动驾驶}
如果可以在各种环境中稳健地正确识别行驶区域的话,实现自动驾驶可
能也就没那么遥远了。最近,在识别周围环境的技术中,深度学习的力量备
受期待。比如,基于CNN的神经网络SegNet。
\subsection{Deep Q-Network(强化学习)}
就像人类通过摸索试验来学习一样(比如骑自行车),让计算机也在摸索
试验的过程中自主学习,这称为强化学习(reinforcement learning)。

强化学习的基本框架是,代理(Agent)根据环境选择行动,然后通过这
个行动改变环境。根据环境的变化,代理获得某种报酬。强化学习的目的是
决定代理的行动方针,以获得更好的报酬。这里需要注意的是,报酬并不是
确定的,只是“预期报酬”。

在使用了深度学习的强化学习方法中,有一个叫作Deep Q-Network(通
称DQN\marginpar[DQN]{DQN})的方法。该方法基于被称为Q学习的强化学习算法。在Q学习中,为了确定最合适的行动,需要确定一个被
称为最优行动价值函数的函数。为了近似这个函数,DQN使用了深度学习
(CNN)。