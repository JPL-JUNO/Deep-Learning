\chapter{基本概念}
\section{函数}
\section{导数}
\section{嵌套函数}
\section{链式法则}
\section{示例介绍}
\subsection*{数学}
从数学上讲,对于包含 3 个基本可微的函数的复合函数,其导数的计算公式如下:
\begin{equation*}
    \frac{df_3}{du}(x)=\frac{df_3}{du}(f_2(f_1(x)))\times\frac{df_2}{du}(f_1(x))\times\frac{df_1}{du}(x)
\end{equation*}
\subsection*{示意图}
要理解以上公式,最为直观的方法就是通过盒子示意图,如 \autoref{The box model for computing the derivative of three nested functions} 所示。
\figures{The box model for computing the derivative of three nested functions}
\subsection*{}
注意,在计算这个嵌套函数的链式法则时,这里对它进行了两次“传递”。
\begin{enumerate}
    \item “向前”传递它,计算出 \verb|f1_of_x| 和 \verb|f2_of_x|,这个过程可以称作(或视作)“前向传
          递”。
    \item “向后”通过函数,使用在前向传递中计算出的量来计算构成导数的量。
          最后,将这 3 个量相乘,得到导数。
\end{enumerate}
\section{多输入函数}
\subsection*{示意图}
既然谈到了多输入函数,现在来定义我们一直在讨论的一个概念:用箭头表示数学运算顺
序的示意图叫作\textbf{计算图}(computational graph)。

\figures{Function with multiple inputs}

可以看到,两个输入进入 $\alpha$ 输出 $a$,然后 $a$ 再被传递给了 $\sigma$ 。
\section{多输入函数的导数}
\section{多向量输入函数}
\subsection*{数学}
在神经网络中,表示单个数据点的典型方法是将 $n$ 个特征列为一行,其中每个特征都只是
一个数字,如
$x_1$ 、$x_2$ 等表示如下:
$$\bm{X}=[x_1, x_2, \cdots, x_n]$$
\section{基于已有特征创建新特征}
神经网络中最常见的运算也许就是计算已有特征的加权和,加权和可以强化某些特征而弱
化其他特征,从而形成一种新特征,但它本身仅仅是旧特征的组合。用数学上的一种简洁
的方式表达就是使用该观测值的点积(dot product),配合与特征
$w_1, w_2, \dots, w_n$
等长的一
组权重。

\subsection*{数学}
存在$\bm{W}=[w_1, w_2, \dots, w_n]^T$,那么可以将此运算的输出定义为:
$$N=v(\bm{X}, \bm{W})=\bm{X}\bm{W}=x_1w_1+x_2w_2+\dots+x_nw_n$$
\section{多向量输入函数的导数}
如果将点积写为$v(\bm{X}, \bm{W})=N$
这种形式,那么自然会产生一个问题:$\frac{\partial N}{\partial \bm{X}}$和$\frac{\partial N}{\partial \bm{W}}$分别是什么?
\subsection*{数学}
矩阵语法只是对一堆以特定形式排列的数字的简
写,“矩阵的导数”实际上是指“矩阵中每个元素的导数”。由于 $\bm{X}$ 有一行,因此它可以这
样定义:
\begin{equation*}
    \frac{\partial v}{\partial \bm{X}}=
    \left[
        \frac{\partial v}{\partial x_1}  \frac{\partial v}{\partial x_2}  \frac{\partial v}{\partial x_3}
        \right]
\end{equation*}

又有$$N=v(\bm{X}, \bm{W})=\bm{X}\bm{W}=x_1w_1+x_2w_2+\dots+x_nw_n$$

因此可以得出:
\begin{equation*}
    \begin{aligned}
        \frac{\partial v}{\partial x_1}    & =w_1, \frac{\partial v}{\partial x_2}  =w_2, \frac{\partial v}{\partial x_3}=w_3 \\
        \frac{\partial v}{\partial \bm{X}} & =
        \left[
            w_1, w_2, w_3
        \right]=\bm{W}^T                                                                                                      \\
    \end{aligned}
\end{equation*}

这个结果出乎意料地简练,掌握这一点极为关键,既可以理解深度学习的有效性,又可以
知道如何清晰地实现深度学习。

以此类推,可以得到如下公式:
\begin{equation*}
    \frac{\partial v}{\partial \bm{W}}=
    \left[
        \begin{matrix}
            x_1 \\
            x_2 \\
            x_3 \\
        \end{matrix}
        \right]=\bm{X}^T
\end{equation*}

\subsection*{代码}
这里计算的 \verb|dNdX| 表示 $\bm{X}$ 的每个元素相对于输出 $N$ 的和的偏导数。在本书中,这个量有一
个特殊的名称,即 $\bm{X}$ 的梯度(gradient)。这个概念是指,对于 $\bm{X}$ 的单个元素(例如
$x_3$ ),
\verb|dNdX| 中的对应元素(具体来说是 \verb|dNdX[2]|)是向量点积 $N$ 的输出相对于
$x+3$ 的偏导数。
\section{向量函数及其导数:再进一步}
假设函数接受向量$\bm{X}$ 和向量 $\bm{W}$,执行 点积(将其表示为 $v(\bm{X}, \bm{W}) $,然后将向量输入到函数$\sigma$ 中。
\subsection*{数学}
公式很简单,如下所示:
$$s = f (\bm{X}, \bm{W}) = \sigma( v(\bm{X}, \bm{W}))= \sigma(x_1w_1 + x_2  w_2 + x_3w_3)$$

\subsection*{向量函数及其导数:后向传递}
\subsubsection*{数学}
由于$f(\bm{X}, \bm{W})=\sigma(v(\bm{X}, \bm{W}))$,因此该函数在$\bm{X}$出的导数可以这样表示:
\begin{equation*}
    \begin{aligned}
        \frac{\partial f}{\partial \bm{X}} & =\frac{\partial \sigma}{\partial u}(v(\bm{X}, \bm{W})) \frac{\partial v}{\partial \bm{X}}(\bm{X},\bm{W}) \\
                                           & =\frac{\partial \sigma}{\partial u}(x_1w_1 + x_2  w_2 + x_3w_3)\bm{W}^T                                  \\
    \end{aligned}
\end{equation*}
\section{包含两个二维矩阵输入的计算图(更一般、更实际的情况)}
在深度学习和更通用的机器学习中,需要处理输入为两个二维数组的运算,其中一个数组
表示一批数据 $\bm{X}$,另一个表示权重 $\bm{W}$。
\subsection*{数学}
假设$\bm{X}$和$\bm{W}$如下所示:
\begin{equation*}
    \begin{aligned}
        \bm{X} & =\left[\begin{matrix}
                                x_{11} & x_{12} & x_{13} \\
                                x_{21} & x_{22} & x_{23} \\
                                x_{31} & x_{32} & x_{33} \\
                            \end{matrix} \right] \\
        \bm{W} & =\left[ \begin{matrix}
                                 w_{11} & w_{12} \\
                                 w_{21} & w_{22} \\
                                 w_{31} & w_{32} \\
                             \end{matrix}  \right]   \\
    \end{aligned}
\end{equation*}
这可能对应一个数据集,其中每个观测值都具有 3 个特征,即矩阵的列数对应特征数,3 行可能对应要对其进行预测
的 3 个观测值,即矩阵的行数对应观测值。

现在将为这些矩阵定义以下简单的运算。
\begin{enumerate}
    \item 将这些矩阵相乘。和以前一样,将把执行此运算的函数表示为 $v(\bm{X}, \bm{W})$,将输出表示为$N$。
    \item  将结果$N$ 传递给可微函数 $\sigma$ ,并定义$S=\sigma(N)$。
\end{enumerate}

现在的问题是:输出 S 相对于 X 和 W 的梯度是多少?

这就引出了一个微妙但十分重要的概念:可以在目标多维数组上执行任何一系列运算,但
是要对某些输出定义好梯度,这需要对序列中的最后一个数组求和(或以其他方式聚合成
单个数字),这样“X 中每个元素的变化会在多大程度上影响输出”这一问题才有意义。

\begin{equation*}
    \begin{aligned}
        \bm{X}\bm{W} & =
        \left[
            \begin{matrix}
                x_{11}w_{11}+x_{12}w_{21}+x_{13}w_{31} & x_{11}w_{12}+x_{12}w_{22}+x_{13}w_{32} \\
                x_{21}w_{11}+x_{22}w_{21}+x_{23}w_{31} & x_{21}w_{12}+x_{22}w_{22}+x_{23}w_{32} \\
                x_{31}w_{11}+x_{32}w_{21}+x_{33}w_{31} & x_{31}w_{12}+x_{32}w_{22}+x_{33}w_{32} \\
            \end{matrix}
        \right]                \\
                     & =\left[
            \begin{matrix}
                XW_{11} & XW_{12} \\
                XW_{21} & XW_{22} \\
                XW_{31} & XW_{32} \\
            \end{matrix}
        \right]                \\
    \end{aligned}
\end{equation*}
为了便于书写结果矩阵,这里将第 $i$ 行的第 $j$ 列表示为
$XW_{ij}$。

接下来,将该结果输入到 $\sigma$ 中:
\begin{equation*}
    \sigma(\bm{X}\bm{W}) =
    \left[
        \begin{matrix}
            \sigma(XW_{11}) & \sigma(XW_{12}) \\
            \sigma(XW_{21}) & \sigma(XW_{22}) \\
            \sigma(XW_{31}) & \sigma(XW_{32}) \\
        \end{matrix}
        \right]
\end{equation*}

最后,对这些元素求和:
\begin{equation*}
    \begin{aligned}
        L & =\Lambda(\sigma(\bm{X}\bm{W}))=\Lambda\left(\left[
            \begin{matrix}
                \sigma(XW_{11}) & \sigma(XW_{12}) \\
                \sigma(XW_{21}) & \sigma(XW_{22}) \\
                \sigma(XW_{31}) & \sigma(XW_{32}) \\
            \end{matrix}
        \right]\right)                                                                                       \\
          & =\sigma(XW_{11})+\sigma(XW_{12})+\sigma(XW_{21})+\sigma(XW_{22})+\sigma(XW_{31})+\sigma(XW_{32}) \\
    \end{aligned}
\end{equation*}

现在回到了纯微积分的场景中:存在一个数字 $L$,想计算出 $L$ 相对于 $\bm{X}$ 和 $\bm{W}$ 的梯度,也
就是明确这些输入矩阵中每个元素的变化对 $L$ 的影响。可以这样写:

\begin{equation*}
    \frac{\partial\Lambda}{\partial u}=\left[
        \begin{matrix}
            \frac{\partial\Lambda}{\partial u}(x_{11}) & \frac{\partial\Lambda}{\partial u}(x_{12}) & \frac{\partial\Lambda}{\partial u}(x_{13}) \\
            \frac{\partial\Lambda}{\partial u}(x_{21}) & \frac{\partial\Lambda}{\partial u}(x_{22}) & \frac{\partial\Lambda}{\partial u}(x_{23}) \\
            \frac{\partial\Lambda}{\partial u}(x_{31}) & \frac{\partial\Lambda}{\partial u}(x_{32}) & \frac{\partial\Lambda}{\partial u}(x_{33}) \\
        \end{matrix}
        \right]
\end{equation*}
\section{有趣的部分:向后传递}
\subsection*{数学}
值 $L$ 实际上是$x_{11}, x_{12}, x_{13}, x_{21}, \cdots, x_{33}$的一个函数。

矩阵显著地分解成:

\begin{equation*}
    \frac{\partial \Lambda}{\partial u}(\bm{X})=\frac{\partial \Lambda}{\partial u}(S)\times\frac{\partial \sigma}{\partial u}(N)\bm{W}^T
\end{equation*}

其中前两个是逐个元素执行的,第三个则是矩阵乘法。

\figures{Graph with a matrix multiplication-the backward pass}

$L$ 相对于 $\bm{W}$ 的梯度的表达式为 $\bm{X}^T$
。但是,$\bm{X}^T$ 表达式中的因子是从 $L$ 的导数中
导出的,考虑到它们的顺序,$\bm{X}^T$ 将位于 $L$ 相对于 $\bm{W}$ 的梯度的表达式的左侧:
\begin{equation*}
    \frac{\partial \Lambda}{\partial u}(\bm{W})=\bm{X}^T\frac{\partial \Lambda}{\partial u}(S)\times\frac{\partial \sigma}{\partial u}(N)
\end{equation*}